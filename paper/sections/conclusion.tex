\section{conclusion}
In this study, a deep convolutional network optimized by a genetic algorithm was proposed to detect hand movements using forearm electromyographic signals. It was observed that although deep networks have a high potential in identifying the EMG patterns, they will not perform well unless their structure is appropriately arranged. In this work, only three parameters, the number of convolutional layers, the number of kernels in each convolutional layer, and the number of dense layer neurons, have been optimized. Finding the optimal value for these parameters increased the model accuracy from 91.86\% to 96.4\% at best, and 95.3\% on average in real-time usage. The accuracy of the optimized model in offline mode is 99.6\%. Optimizing the number of epochs and training the model up to 100\% convergence point can significantly affect the model's accuracy. The objective function of the genetic algorithm in this research is the accuracy of the model. To reduce the computational load, adding the number of independent model variables to the objective function's output can prevent the optimizer from moving toward larger models. In this case, a compromise is made between the accuracy of the model and the computational load. The authors focus on overcoming computational constraints and implementing new initiatives in model architecture and optimization in future work.