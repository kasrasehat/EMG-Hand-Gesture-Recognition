% HELP
% bold: {\bf The deadline for the submission of a manuscript to the CR is 1 December 2021.}
% \textit{IEEE Transactions on Nuclear Science (TNS)}

\section{Introduction}
\IEEEPARstart{T}{he} EMG signal directly results from the muscles' electrical activity during contraction and can be considered a means of decoding body movements [1]. Human motion detection using EMG signals, known as EMG pattern recognition, has been employed in various applications, including powered upper-limb prostheses [2], electric power wheelchairs [3], human-computer interactions [4], and diagnosis in clinical applications [5]. Compared to other well-known bioelectrical signals (e.g., electrocardiogram) ECG), the electrooculogram (EOG), and galvanic skin response (GSR)), the analysis of surface EMG signal is more challenging [6]. Some of these challenges include the changing characteristics of the signal itself over time, electrode location shift, muscle fatigue, variations in muscle contraction intensity, limb position changes, and forearm orientation. However, due to its non-invasive nature and ease of recording sEMG signals, its use is superior to other methods [7-12].

The most common EMG pattern recognition algorithms include support vector machine (SVM), K nearest neighbor (K-NN), Linear discriminant analysis (LDA), multilayer perceptron (MLP), artificial neural network (ANN), and random forest classifiers. Ahmad Alkan and Mucahid Gunay [14] have claimed that they could classify four EMG data classes with 97±1\% accuracy using the SVM algorithm. Tkach et al. [15] used the LDA algorithm to classify the data. They achieved 92\% accuracy for 13 classes. Z. Li et al. [16], using a boosted random forest classifier, achieved a RR of 92\%, but the novelty detection accuracy was 20\%. The authors tuned the algorithm so that there is a novel detection accuracy of 80\%, but the RR in trained samples decreased to 80\%. In the mentioned methods, the step before classification is feature extraction, which is done manually.

Although feature engineering has been the dominant focus for EMG pattern recognition so far, feature learning, as exemplified by deep learning, has recently started to demonstrate better recognition performance than hand-crafted features. Unlike feature engineering and conventional machine learning approaches, deep learning can take advantage of the multiplicity of many samples to extract high-level features from low-level inputs. However, deep learning algorithms require large training datasets to train large deep networks (a few hidden layers, each with a large number of neurons) and an associated large number of parameters (millions of free parameters). Nowadays, despite large datasets and advances in processing technology, deep learning algorithms in EMG pattern recognition are less restricted [17].

Geng et al. [18] evaluated CNN's performance in recognizing hand and finger motions based on sEMG from three public databases containing data recorded from either a single row of electrodes or a 2D high-density electrode array. Without using windowed features, the classification accuracy of an eight-motion within-subject problem achieved 89.3\% on a single frame (1 ms) of an sEMG image and reached 99.0\% and 99.5\% using 128 channels HD-sEMG signals and simple majority voting over 40 and 150 frames (40 and 150 ms), respectively. Côté-Allard et al. [19,20] showed that CNN is accurate enough to detect complex movements. It is also robust to short-term muscle fatigue, small displacement of electrodes, and long-term use without recalibration. Laezza [21] evaluated three different network models' performance, including RNN, CNN, and RNN + CNN for myoelectric control. Based on their work, RNN provided the best performance with 91.81\%  classification accuracy compared to CNN (89.01\%) and RNN + CNN (90.4\%). This result may be due to RNN and LSTM networks' advantages in time series processing. Ali Raza Asif et al. [22] investigated the effect of hyperparameters on each hand gesture. They showed that the learning rate set to either 0.0001 or 0.001 significantly outperformed other considerations.

In the present study, the UC2018 database [23] has been used to train a deep convolutional neural network. Since the dataset does not have enough data for deep network training, more data has been generated using the windowing technique [24]. First, the basic model, a deep convolutional network, is trained by the data obtained. Due to a large number of variables, the initial model does not necessarily have the maximum accuracy. Therefore, to achieve maximum accuracy, the model structure is optimized by a genetic algorithm [25]. Finally, the model confusion matrix is proposed using the optimal parameters, representing the designed model's performance.
