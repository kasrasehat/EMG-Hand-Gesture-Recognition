\section{Model Optimization}
In the previous section, a model was used to recognize hand gesture based on the EMG signal. However, this model includes parameters that a slight change in each affects the accuracy of the model. Therefore, it is essential to select these parameters so that the model is most accurate. So we are faced with an optimization problem, and the cost function is the accuracy of the convolutional model in detecting hand gesture. One way to get the optimal answer is to experiment with different structure of models and choose the best answer. However, this is time-consuming, and a limited number of models can be tested. The optimal answer obtained in this way is probably a local minimum point and not the best. A genetic algorithm is a Stochastic optimization method and uses random elements, so it is improbable to be trapped at local minimum points. As the rate of random parameters increases, this probability decreases.

The genetic algorithm determines the first generation with a specific population size so that each of the chromosomes of this generation is a structure for the convolutional model. The convolutional model is based on chromosomes' structure and provides recogntion accuracy, which is the algorithm's objective function. The chromosomes are then randomly integrated to form the next generation. The higher the chromosome score from the objective function, the better the chance of advancing to the next generation. Chromosome integration is done by cross over, mutation, parents portion, and elite portion methods [36]. This process continues until the algorithm converges and reaches the best generation. The best chromosome of this generation is selected as the optimal structure of the model.

The genetic algorithm has parameters that must be adjusted for better results. One of these parameters is the number of chromosomes in each generation, between 2 and 4 times the number of problem variables. In this case, the number of variables is 4, but for a better result, chromosomes' population is considered to be 25. The maximum number of generation sequences is a large number selected not to cause the algorithm to stop before convergence. In the crossover method, two members of the current population are randomly selected with different weight coefficients. The cross over probability coefficient is 0.7 and is selected as uniform. Each member of the new generation will be generated with a 20\% chance of crossing over. The mutation probability coefficient in this study is 0.1. The larger this number, the less likely the algorithm be trapped in a local minimum. In the parent's portion method, some parents are passed on to the next generation randomly but with different weight coefficients. The probability of choosing parents is based on the score they have earned from the objective function. The parent's portion coefficient is 0.2.  In the elite method, some elites of each generation are passed on to future generations. The share of elites from each generation is 0.05. A stop criterion is also defined to stop if the algorithm converges before reaching max iteration. If there is no increase in the objective function in 6 consecutive generations, it means that it has reached the convergence point.

The objective function of the genetic algorithm is the accuracy of the model. Therefore, for each generation member, the model must be trained, and the accuracy of the model must be determined, which makes the computational cost very high. A method has been used to reduce the computational cost as much as possible. A filter is placed in the objective function whose job is to detect duplicate inputs. If the input is duplicated and the model has already been trained with the same structure, the algorithm searches for the corresponding answer among the previous answers and records it as the objective function's output.


The research algorithm is written in Python programming language on the Tensorflow framework and the Keras library. The test runs on a computer with a 3.2GHz CPU, 32GB RAM \&  NVIDIA GeForce GTX 1080 specifications.
